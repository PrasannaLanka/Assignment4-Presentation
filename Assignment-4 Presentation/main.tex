\documentclass{beamer}
\usepackage{listings}
\lstset{
%language=C,
frame=single, 
breaklines=true,
columns=fullflexible
}
\usepackage{blkarray}
\usepackage{subcaption}
\usepackage{url}
\usepackage{tikz}
\usepackage{tkz-euclide} % loads  TikZ and tkz-base
%\usetkzobj{all}
\usetikzlibrary{calc,math}
\usepackage{float}
\newcommand\norm[1]{\left\lVert#1\right\rVert}
\renewcommand{\vec}[1]{\mathbf{#1}}
\usepackage[export]{adjustbox}
\usepackage[utf8]{inputenc}
\usepackage{amsmath}
\usepackage{tikz}
\usetikzlibrary{automata, positioning}
\usetheme{Boadilla}
\providecommand{\pr}[1]{\ensuremath{\Pr\left(#1\right)}}

\title{CSIR UGC NET EXAM (June 2013), Q.84}
\author{Lanka Prasanna}
\date{CS20BTECH11029}
\begin{document}

\begin{frame}
\titlepage
\end{frame}

\begin{frame}
\frametitle{}
\begin{block}{Uniform Distribution}
A random variable $X$ is said to be uniformly distributed in $a\leq x\leq b$ if its density function is
\begin{align}
    f(x)=
    \begin{cases}
    \frac{1}{b-a} & \text{if } a\leq x \leq b\\
    0 & \text{otherwise}
    \end{cases}\label{eq:1}
\end{align}
and the distribution is called uniform distribution.
The mean and variance are respectively,
\begin{align}
    \mu=\frac{a+b}{2}\label{eq:2}
\end{align}
\begin{align}
     \sigma^2=\frac{(b-a)^2}{12}\label{eq:3}
\end{align}
\label{theorem}

\end{block}
\end{frame}

\begin{frame}
\frametitle{}
\begin{block}{Beta distribution}
The Beta distribution is a continuous distribution defined on the range
(0,1) where the parameters are given by\\
If $X\sim B(r,s)$, where $B(r,s)$ is a beta function
\begin{align}
  \label{eq:4}  f(x)&=\frac{1}{B(r,s)}x^{r-1}(1-x)^{s-1}\\ 
    \label{eq:5}F(x)&=\int_{0}^{x}\frac{1}{B(r,s)}x^{r-1}(1-x)^{s-1}dx=\frac{B_x(r,s)}{B(r,s)}\\ 
  \label{eq:6}  B(r,s)&=\int_{0}^{1}x^{r-1}(1-x)^{s-1}dx=\frac{(r-1)!(s-1)!}{(r+s-1)!}\\ 
     \label{eq:7}B_x(r,s)&=\int_{0}^{x}x^{r-1}(1-x)^{s-1}dx\\
   \label{eq:8} E(X)&=\frac{r}{r+s}\\ 
    \label{var} Var(X)&=\frac{rs}{(r+s)^{2}(r+s+1)} 
\end{align}
\end{block}
\end{frame}

\begin{frame}
\frametitle{}
\begin{block}{Order statistics}
For given statistical sample $\{X_1, X_2,\cdots X_n\}$, the order statistics is obtained by sorting the sample in ascending order. It denoted as $\{X_{(1)}, X_{(2)},\cdots X_{(n)}\}$.
\end{block}

\frametitle{}
\begin{block}{Median of order statistics}
Median is defined as the middle number of a sorted sample. It is denoted by M and defined using order statistics of a sample as
\begin{align}
  M =
  \begin{cases}
   X_{((n+1)/2)},                                           &\text{if $n$ is odd,} \\ \\
  \dfrac{ X_{(n/2)} + X_{(n/2+1)}}{2} ,                     &\text{if $n$ is even,} 
  \end{cases}
\end{align}
\label{median}\label{def2}
\end{block}
\end{frame}

\begin{frame}
\frametitle{}
\begin{block}{Note}
The order statistics of the uniform distribution on the unit interval have marginal distributions belonging to the Beta distribution family.
\begin{align}
X_{(k)} \sim B(k,n+1-k)
\end{align}\label{rem}
\end{block}
\end{frame}



\begin{frame}
\frametitle{Question}
\begin{block}{CSIR UGC NET EXAM (June 2013), Q.84}
Let $X_1,X_2,X_3,X_4,X_5$ be independent and identically distributed random variables each following a uniform distribution on (0,1) and M denote their median. Then which of the following statements are true?
\begin{enumerate}
    \item $\pr{M<\frac{1}{3}}=\pr{M>\frac{2}{3}}$\\
    \item $M$ is uniformly distributed on (0,1)\\
    \item $E(M)=E(X_1)$\\
    \item $V(M)=V(X_1)$
\end{enumerate}
\end{block}
\end{frame}

\begin{frame}
\frametitle{Solution}
The order statistics of given sample is $\{X_{(1)},X_{(2)},X_{(3)},X_{(4)},X_{(5)}\}$.\\
From definition \eqref{median} median $M$ is given by
\begin{align}
  M &= X_{((5+1)/2)}\\
  &=X_{(3)}\label{eq:m}
\end{align}
From note \eqref{rem} 
\begin{align}
X_{(3)} \sim B(3,3)
\end{align}
 \end{frame}

\begin{frame}
\frametitle{Option 1}
From equation \eqref{eq:6}, Beta function of median is given by
\begin{align}
    B(3,3)&=\frac{(3-1)!(3-1)!}{(3+3-1)!}=\frac{1}{30}
 \end{align}
 From equation \eqref{eq:4}, PDF of median is
 \begin{align}
     f(x)=30x^{2}(1-x)^{2}
 \end{align}
 From equation \eqref{eq:5}, CDF of median is 
 \begin{align}
     F(x)&=\int_{0}^{x}30x^{2}(1-x)^{2}dx\\
     &=30x^{3}\left(\frac{1}{3}+\frac{x^2}{5}-\frac{x}{2}\right)
 \end{align}
 \end{frame}
 \begin{frame}
 \begin{align}
     \pr{M<\frac{1}{3}}&=F\left(\frac{1}{3}\right)\\
     &=0.20987\\
     \pr{M>\frac{2}{3}}&=\pr{M<1}-\pr{M<\frac{2}{3}}\\
     &=F(1)-F\left(\frac{2}{3}\right)\\
     &=0.20987
      \end{align}
 \begin{align}
      \therefore \pr{M<\frac{1}{3}}=\pr{M>\frac{2}{3}}
 \end{align}
  Hence \textbf{Option 1 is true.}
\end{frame}
\begin{frame}
\frametitle{Option 2}
From \eqref{eq:m}, median $M$ is a third order statistic. Clearly from note \eqref{rem}, $M$ is a Beta distribution whose PDF is given by
\begin{align}
    f(x)=
    \begin{cases}
    30x^{2}(1-x)^{2} & 0 \leq x \leq 1\\
    0 & \text{otherwise}
    \end{cases}\label{eq:25}
\end{align}
which is not a uniform distribution.
\\Hence \textbf{Option 2 is false.}
    
\end{frame}
\begin{frame}{Option 3}
    From equation \eqref{eq:1}, density function for each of given random variables $X_1,X_2,X_3,X_4,X_5$ is given by
\begin{align}
    f(x)=
    \begin{cases}
    1 & \text{if } 0\leq x \leq 1\\
    0 & \text{otherwise}
    \end{cases}
\end{align}
From equation \eqref{eq:2} 
\begin{align}
    E(X_1)&=\frac{1}{2}
\end{align}
From equation \eqref{eq:8}
\begin{align}
    E(M)=\frac{3}{3+3}=\frac{1}{2}
\end{align}
 \begin{align}
      \therefore E(M)=E(X_1)
 \end{align}
   Hence \textbf{Option 3 is true.}
\end{frame}
\begin{frame}{Option 4}
From equation \eqref{eq:3}
\begin{align}
      V(X_1)&=\frac{1}{12}
\end{align}
From equation \eqref{var}
\begin{align}
    V(M)=\frac{1}{28}
\end{align}
\begin{align}
      \therefore V(M)\neq V(X_1)
 \end{align}
   Hence \textbf{Option 4 is false.}
    
\end{frame}


\end{document}