\documentclass[journal,12pt,twocolumn]{IEEEtran}
\usepackage[shortlabels]{enumitem}
\usepackage{setspace}
\usepackage{gensymb}
\singlespacing
\usepackage[cmex10]{amsmath}
\usepackage{graphicx}

\usepackage{float}
\usepackage{amsthm}

\usepackage{mathrsfs}
\usepackage{txfonts}
\usepackage{stfloats}
\usepackage{bm}
\usepackage{cite}
\usepackage{cases}
\usepackage{subfig}

\usepackage{longtable}
\usepackage{multirow}

\usepackage{enumitem}
\usepackage{mathtools}
\usepackage{steinmetz}
\usepackage{tikz}
\usepackage{circuitikz}
\usepackage{verbatim}
\usepackage{tfrupee}
\usepackage[breaklinks=true]{hyperref}
\usepackage{graphicx}
\usepackage{tkz-euclide}
\newtheorem{definition}{Definition}[section]
\usetikzlibrary{calc,math}
\usepackage{listings}
    \usepackage{color}                                            %%
    \usepackage{array}                                            %%
    \usepackage{longtable}                                        %%
    \usepackage{calc}                                             %%
    \usepackage{multirow}                                         %%
    \usepackage{hhline}                                           %%
    \usepackage{ifthen}                                           %%
    \usepackage{lscape}     
\usepackage{multicol}
\usepackage{amsmath}
\usepackage{chngcntr}
\usepackage{algorithmic}
\usepackage{autobreak}
\allowdisplaybreaks
\DeclareMathOperator*{\Res}{Res}

\renewcommand\thesection{\arabic{section}}
\renewcommand\thesubsection{\thesection.\arabic{subsection}}
\renewcommand\thesubsubsection{\thesubsection.\arabic{subsubsection}}

\renewcommand\thesectiondis{\arabic{section}}
\renewcommand\thesubsectiondis{\thesectiondis.\arabic{subsection}}
\renewcommand\thesubsubsectiondis{\thesubsectiondis.\arabic{subsubsection}}


\hyphenation{op-tical net-works semi-conduc-tor}
\def\inputGnumericTable{}  %%
\newtheorem{theorem}{Theorem}[section]
\newtheorem{defn}[theorem]{Definition}
\newtheorem{lemma}[theorem]{Lemma}
\newtheorem{remark}[theorem]{Remark}
\lstset{
%language=C,
frame=single, 
breaklines=true,
columns=fullflexible
}
\begin{document}

\newcommand{\BEQA}{\begin{eqnarray}}
\newcommand{\EEQA}{\end{eqnarray}}
\newcommand{\define}{\stackrel{\triangle}{=}}
\bibliographystyle{IEEEtran}
\raggedbottom
\setlength{\parindent}{0pt}
\providecommand{\mbf}{\mathbf}
\providecommand{\pr}[1]{\ensuremath{\Pr\left(#1\right)}}
\providecommand{\qfunc}[1]{\ensuremath{Q\left(#1\right)}}
\providecommand{\sbrak}[1]{\ensuremath{{}\left[#1\right]}}
\providecommand{\lsbrak}[1]{\ensuremath{{}\left[#1\right.}}
\providecommand{\rsbrak}[1]{\ensuremath{{}\left.#1\right]}}
\providecommand{\brak}[1]{\ensuremath{\left(#1\right)}}
\providecommand{\lbrak}[1]{\ensuremath{\left(#1\right.}}
\providecommand{\rbrak}[1]{\ensuremath{\left.#1\right)}}
\providecommand{\cbrak}[1]{\ensuremath{\left\{#1\right\}}}
\providecommand{\lcbrak}[1]{\ensuremath{\left\{#1\right.}}
\providecommand{\rcbrak}[1]{\ensuremath{\left.#1\right\}}}
\theoremstyle{remark}
\newtheorem{rem}{Remark}
\newcommand{\sgn}{\mathop{\mathrm{sgn}}}
\providecommand{\abs}[1]{\vert#1\vert}
\providecommand{\res}[1]{\Res\displaylimits_{#1}} 
\providecommand{\norm}[1]{\lVert#1\rVert}
%\providecommand{\norm}[1]{\lVert#1\rVert}
\providecommand{\mtx}[1]{\mathbf{#1}}
\providecommand{\mean}[1]{E[ #1 ]}
\providecommand{\fourier}{\overset{\mathcal{F}}{ \rightleftharpoons}}
%\providecommand{\hilbert}{\overset{\mathcal{H}}{ \rightleftharpoons}}
\providecommand{\system}{\overset{\mathcal{H}}{ \longleftrightarrow}}
	%\newcommand{\solution}[2]{\textbf{Solution:}{#1}}
\newcommand{\solution}{\noindent \textbf{Solution: }}
\newcommand{\cosec}{\,\text{cosec}\,}
\providecommand{\dec}[2]{\ensuremath{\overset{#1}{\underset{#2}{\gtrless}}}}
\newcommand{\myvec}[1]{\ensuremath{\begin{pmatrix}#1\end{pmatrix}}}
\newcommand{\mydet}[1]{\ensuremath{\begin{vmatrix}#1\end{vmatrix}}}
\numberwithin{equation}{subsection}
\makeatletter
\@addtoreset{figure}{problem}
\makeatother
\let\StandardTheFigure\thefigure
\let\vec\mathbf
\renewcommand{\thefigure}{\theproblem}
\def\putbox#1#2#3{\makebox[0in][l]{\makebox[#1][l]{}\raisebox{\baselineskip}[0in][0in]{\raisebox{#2}[0in][0in]{#3}}}}
     \def\rightbox#1{\makebox[0in][r]{#1}}
     \def\centbox#1{\makebox[0in]{#1}}
     \def\topbox#1{\raisebox{-\baselineskip}[0in][0in]{#1}}
     \def\midbox#1{\raisebox{-0.5\baselineskip}[0in][0in]{#1}}
\vspace{3cm}
\title{Assignment-4}
\author{Lanka Prasanna-CS20BTECH11029}
\maketitle
\newpage
\bigskip
\renewcommand{\thefigure}{\theenumi}
\renewcommand{\thetable}{\theenumi}
\text{Download all latex codes from:}
\begin{lstlisting}
https://github.com/PrasannaLanka/Assignment4/blob/main/Assignment4/codes/Assignment4.tex
\end{lstlisting}
and python codes from
\begin{lstlisting}
https://github.com/PrasannaLanka/Assignment4/blob/main/Assignment4/codes/Assignment4.py
\end{lstlisting}
\section*{Problem: CSIR UGC NET EXAM (June 2013), Q.84}
Let $X_1,X_2,X_3,X_4,X_5$ be independent and identically distributed random variables each following a uniform distribution on (0,1) and M denote their median. Then which of the following statements are true?
\begin{enumerate}
    \item $\pr{M<\frac{1}{3}}=\pr{M>\frac{2}{3}}$\\
    \item $M$ is uniformly distributed on (0,1)\\
    \item $E(M)=E(X_1)$\\
    \item $V(M)=V(X_1)$
\end{enumerate}
\section*{solution}


\begin{theorem}[Uniform distribution]
A random variable $X$ is said to be uniformly distributed in $a\leq x\leq b$ if its density function is
\begin{align}
    f(x)=
    \begin{cases}
    \frac{1}{b-a} & \text{if } a\leq x \leq b\\
    0 & \text{otherwise}
    \end{cases}\label{eq:1}
\end{align}
and the distribution is called uniform distribution.
The mean and variance are respectively,
\begin{align}
    \mu=\frac{a+b}{2}\label{eq:2}
\end{align}
\begin{align}
     \sigma^2=\frac{(b-a)^2}{12}\label{eq:3}
\end{align}
\label{theorem}
\end{theorem}

\begin{theorem}[Beta distribution]
The Beta distribution is a continuous distribution defined on the range
(0, 1) where the parameters are given by\\
If $X\sim B(r,s)$, where $B(r,s)$ is a beta function
\begin{align}
  \label{eq:4}  f(x)&=\frac{1}{B(r,s)}x^{r-1}(1-x)^{s-1}\\ 
    \label{eq:5}F(x)&=\int_{0}^{x}\frac{1}{B(r,s)}x^{r-1}(1-x)^{s-1}dx=\frac{B_x(r,s)}{B(r,s)}\\ 
  \label{eq:6}  B(r,s)&=\int_{0}^{1}x^{r-1}(1-x)^{s-1}dx=\frac{(r-1)!(s-1)!}{(r+s-1)!}\\ 
     \label{eq:7}B_x(r,s)&=\int_{0}^{x}x^{r-1}(1-x)^{s-1}dx\\
   \label{eq:8} E(X)&=\frac{r}{r+s}\\ 
    \label{var} Var(X)&=\frac{rs}{(r+s)^{2}(r+s+1)} 
\end{align}
\end{theorem}



\begin{definition}[Order statistics]
For given statistical sample $\{X_1, X_2,\cdots X_n\}$, the order statistics is obtained by sorting the sample in ascending order. It denoted as $\{X_{(1)}, X_{(2)},\cdots X_{(n)}\}$.
\end{definition}


\begin{definition}[Median of order statistics]
Median is defined as the middle number of a sorted sample. It is denoted by M and defined using order statistics of a sample as
\begin{align}
  M =
  \begin{cases}
   X_{((n+1)/2)},                                           &\text{if $n$ is odd,} \\ \\
  \dfrac{ X_{(n/2)} + X_{(n/2+1)}}{2} ,                     &\text{if $n$ is even,} 
  \end{cases}
\end{align}
\label{median}\label{def2}
\end{definition}


\begin{remark}
The order statistics of the uniform distribution on the unit interval have marginal distributions belonging to the Beta distribution family.
\begin{align}
X_{(k)} \sim B(k,n+1-k)
\end{align}\label{rem}
\end{remark}



\begin{enumerate}
 \item 
From definition \eqref{median} median $M$ is given by
\begin{align}
  M &= X_{((5+1)/2)}\\
  &=X_{(3)}\label{eq:m}
\end{align}
From remark \eqref{rem} 
\begin{align}
X_{(3)} \sim B(3,3)
\end{align}
From \eqref{eq:6}
\begin{align}
    B(3,3)&=\frac{(3-1)!(3-1)!}{(3+3-1)!}=\frac{1}{30}
 \end{align}
 From \eqref{eq:4}
 \begin{align}
     f(x)=30x^{2}(1-x)^{2}
 \end{align}
 From \eqref{eq:5}
 \begin{align}
     F(x)&=\int_{0}^{x}30x^{2}(1-x)^{2}dx\\
     &=30x^{3}\brak{\frac{1}{3}+\frac{x^2}{5}-\frac{x}{2}}
 \end{align}
 \begin{align}
     \pr{M<\frac{1}{3}}&=F\brak{\frac{1}{3}}=0.20987\\
     \pr{M>\frac{2}{3}}&=F(1)-F\brak{\frac{2}{3}}=0.20987
 \end{align}
 \begin{align}
      \therefore \pr{M<\frac{1}{3}}=\pr{M>\frac{2}{3}}
 \end{align}
  Hence \textbf{Option 1 is true.}
    
    
    
    
 \item From \eqref{eq:m}, median M is a third order statistic. Clearly from remark \eqref{rem}, $M$ is not an uniform distribution.
\\Hence \textbf{Option 2 is false.}

\item From \eqref{eq:1}
\begin{align}
    f(x)=
    \begin{cases}
    1 & \text{if } 0\leq x \leq 1\\
    0 & \text{otherwise}
    \end{cases}
\end{align}
From \eqref{eq:2} 
\begin{align}
    E(X_1)&=\frac{1}{2}
\end{align}
From \eqref{eq:8}
\begin{align}
    E(M)=\frac{3}{3+3}=\frac{1}{2}
\end{align}
 \begin{align}
      \therefore E(M)=E(X_1)
 \end{align}
   Hence \textbf{Option 3 is true.}

\item From \eqref{eq:3}
\begin{align}
      V(X_1)&=\frac{1}{12}
\end{align}
From \eqref{var}
\begin{align}
    V(M)=\frac{1}{28}
\end{align}
\begin{align}
      \therefore V(M)\neq V(X_1)
 \end{align}
   Hence \textbf{Option 4 is false.}
\end{enumerate}
\end{document}














